%!TEX program = pdflatex
\documentclass{standalone}
\usepackage[UTF8]{ctex}
\usepackage{amsmath}
\usepackage{tikz}
\usepackage{mathdots}
\usepackage{yhmath}
\usepackage{cancel}
\usepackage{color}
\usepackage{siunitx}
\usepackage{array}
\usepackage{multirow}
\usepackage{amssymb}
\usepackage{gensymb}
\usepackage{tabularx}
\usepackage{booktabs}
\usetikzlibrary{fadings}
\tikzset{every picture/.style={line width=0.75pt}} %set default line width to 0.75pt    

\begin{document}
	
\begin{tikzpicture}[x=0.75pt,y=0.75pt,yscale=-1,xscale=1]
%uncomment if require: \path (0,635.1999969482422); %set diagram left start at 0, and has height of 635.1999969482422

%Image [id:dp2637617940860193] 
\draw (240,140) node  {\includegraphics[width=270pt,height=150pt]{Neuron_Hand-tuned.svg}};

% Text Node
\draw (105.5,241) node  [align=left] {细胞核};
% Text Node
\draw (214.5,101) node  [align=left] {细胞体};
% Text Node
\draw (134.5,41) node  [align=left] {树突};
% Text Node
\draw (264.5,221) node  [align=left] {髓鞘};
% Text Node
\draw (257,149) node  [align=left] {轴突};
% Text Node
\draw (293,99) node  [align=left] {兰氏结};
% Text Node
\draw (343,201) node  [align=left] {髓鞘};
% Text Node
\draw (387,59) node  [align=left] {突触};


\end{tikzpicture}


\end{document}
